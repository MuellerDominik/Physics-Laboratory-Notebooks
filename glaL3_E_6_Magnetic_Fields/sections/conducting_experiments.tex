\section{Conducting the Experiments}
\label{sec:Conducting_the_Experiments}
This section describes the experimental arrangement with the device list and all the measurement objects. Furthermore, it contains information about the measuring process.

\subsection{Experimental Arrangement}
\label{subsec:experimental_arrangement}
The experiment was arranged as shown below in figure \ref{fig:experimental_arrangement}. The Hall sensor and the cylindrical coil were placed on a rail. This allowed the coil to be moved along the z-axis.
\begin{figure}[H]
	\centering
	\includegraphics[scale=0.13]{experimental_arrangement}
	\caption{Experimental Arrangement}
	\label{fig:experimental_arrangement}
\end{figure}

The following measurement devices and sensors were used:

\begin{table}[H]
	\centering
	\renewcommand{\arraystretch}{1.2}
	\begin{tabular}{l l l}
		\hline
		\textbf{Device} & \textbf{Designation} & \textbf{Uncertainty} \\
		\hline
		Multimeter & Fluke 179 True RMS & $\pm1$ \% \\
		Gaussmeter & Model 7030 (F.W. Bell) & $\pm0.05$ \% \\
		Hall Sensor & ZOA73-3208-05-T (F.W. BELL) & $\pm0.25$ \% (up to 1 T) \\
		Power Supply & ES 030-5 (Delta Elektronika) & - \\ \hline
	\end{tabular}
	\caption{Measurement Devices and Sensors}
	\label{tab:measurement_devices_and_sensors}
\end{table}

\subsection{Measuring Procedure}
\label{subsec:measuring_procedure}
The magnetic field is measured with a gaussmeter and a Hall sensor from the manufacturer F.W. Bell. Whenever the internal temperature of the gaussmeter changes by $\pm5$ °C, the device must be recalibrated. The earth's magnetic field is compensated by zeroing the gaussmeter after the Hall sensor is placed in the center of the coil. The measurement devices and the power supply were placed at a distance to avoid disturbances.

The power is supplied from a stabilized power supply. The supply lines have been twisted to reduce noise from environmental magnetic fields. To compensate for the losses in the supply lines, the power supply is operated in the constant current mode. The uncertainty of the power supply is irrelevant because the current is measured with the multimeter and is not read from the display on the power supply. The current of 1 A must not be exceeded in order to protect the coil from damage.

\subsection{Measurement Objects}
\label{subsec:measurement_objects}
The measuring objects are two cylindrical coils with copper wire. The main difference between the two coils is their length. All important specifications are listed in the table below:

\begin{table}[H]
	\centering
	\renewcommand{\arraystretch}{1.2}
	\begin{tabular}{r|c|c}
		 & \textbf{Short Cylindrical Coil} & \textbf{Long Cylindrical Coil} \\
		\hline
		\textbf{Inside Diameter $d$} & $(97.0\pm2)$\ mm & $(98.0\pm2)$\ mm \\
		\textbf{Wire Diameter $d_w$} & $0.8$\ mm (Copper) & $0.8$\ mm (Copper) \\
		\textbf{Length $l$} & $(100\pm1)$\ mm & $(200\pm1)$\ mm \\
		\textbf{Turns $n$} & 240 (2 Layers) & 240 (1 Layer) \\ \hline
	\end{tabular}
	\caption{Specifications of the Cylindrical Coils}
	\label{tab:Specifications_Cylindrical_Coils}
\end{table}

\subsection{Measurements}
\label{subsec:Measurements}
The following environmental conditions prevailed during the conduct of the experiments:
\begin{table}[H]
	\centering
	\renewcommand{\arraystretch}{1.2}
	\begin{tabular}{r l}
		\hline
		\textbf{Temperature} & 22.1 °C \\
		\textbf{Air Humidity} & 39 \% \\
		\textbf{Atmospheric Pressure} & 976 hPa \\ \hline
	\end{tabular}
	\caption{Environmental Conditions}
	\label{tab:Environmental_Conditions}
\end{table}
\addtocontents{toc}{\protect\setcounter{tocdepth}{2}}
\subsubsection{Measuring the Magnetic Surface}
\label{subsubsec:Measuring_the_Magnetic_Surface}
The measurements of the magnetic field are influenced by the environment. Zeroing the gaussmeter only compensates for these conditions at a certain point and not over the entire measuring range. The Hall sensor is used to measure the magnetic surface on several points along the z-axis.

\subsubsection{Measuring the Central Value $B_0(I)$}
\label{subsubsec:Measuring_the_Central_Value}
The Hall sensor is first placed in the center of the coil. Then, the magnetic flux density $B_0$ (central value) is measured as a function of the current ranginf from 0 A to 1 A.

\subsubsection{Measuring the Field Pattern $B_z(z)$}
\addtocontents{toc}{\protect\setcounter{tocdepth}{3}}
\label{subsubsec:Measuring_the_Field_Pattern}
To measure the magnetic field pattern, the magnetic flux density $B_z(z)$ is measured at several positions along the z-axis. To perform the measurements the cylindrical coil is moved along the z-axis while the Hall sensor remains stationary. The current through the coil is set to 1 A and kept constant. The distance between individual measurements depends on how fast the magnetic field changes. The faster it changes, the smaller the distance between measurements and the more measurements are performed.
