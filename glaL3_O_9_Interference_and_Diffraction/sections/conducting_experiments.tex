\section{Conducting the Experiments}
\label{sec:Conducting_the_Experiments}
This section describes the experimental arrangement with the device list and all the measurement objects. Furthermore, it contains information about the measuring process.

\subsection{Experimental Arrangement}
\label{subsec:experimental_arrangement}
The experiment was arranged as shown below in figure \ref{fig:experimental_arrangement}:
\begin{figure}[H]
	\centering
	\includegraphics[scale=1]{experimental_arrangement}
	\caption{Experimental Arrangement \cite{interference} - partially modified}
	\label{fig:experimental_arrangement}
\end{figure}
The red laser and the apertures were placed on a rail. This allowed the apertures to be moved alog the x-axis and thus positioned in an ideal spot. The laser with a wavelength of 633 nm then shines through the aperture and the diffraction gets visible on the projection surface. It \textbf{is possible to use a variety of lenses} to optimize the sharpness of the diffraction on the projection surface. In this experiment \textbf{no lenses were used} tough. The diffraction pattern was already sharp enough.

The following equipment was used:

\begin{table}[H]
	\centering
	\renewcommand{\arraystretch}{1.2}
	\begin{tabular}{l l l}
		\hline
		\textbf{Device} & \textbf{Information} & \textbf{Uncertainty} \\
		\hline
		Laser & $\lambda = 633$ nm & N/A \\
		Double Meter & x-Measurements & $\pm1$ mm \\
		Ruler & y-Measurements & $\pm1$ mm \\ \hline
	\end{tabular}
	\caption{Equipment}
	\label{tab:equipment}
\end{table}

The red laser pointer is used to shine through the aperture. The double meter is used to measure the x-distance (alone the x-axis) and the ruler is used to measure the y-position of the maxima and minima. Even tough it is possible to read the position on the ruler to half a millimeter, the uncertainty is still one millimeter due to the difficulty to find the exact center of a minima or maxima.

\subsection{Measuring Procedure}
\label{subsec:measuring_procedure}
First, the aperture is moved along the x-axis to find an ideal spot (sharp diffraction pattern on the projection surface). Then, the x-distance is written down. After that, the absolute value of the center is measured with the ruler. The last step consists of measuring as much absolute y-values of minima or maxima as possible.

\subsection{Measurement Objects}
\label{subsec:measurement_objects}
The measuring objects are a range of different apertures (slits, anti-slits, circular apertures and cross-grids). The different apertures and their specifications are listed in the table below:

\begin{table}[H]
	\centering
	\renewcommand{\arraystretch}{1.2}
	\begin{tabular}{r|c|c|c}
		 & \textbf{Width $w$} & \textbf{Diameter $d$} & \textbf{Grid Constant $g$} \\
		\hline
		\textbf{Slits} & 40 $\mu$m \& 100 $\mu$m & - & - \\
		\textbf{Anti-Slits} & 230 $\mu$m \& 124 $\mu$m & - & - \\
		\textbf{Circular Apertures} & - & 150 $\mu$m \& 100 $\mu$m & - \\
		\textbf{Cross-Grids} & - & - & 28 $\mu$m \& 50 $\mu$m \\ \hline
	\end{tabular}
	\caption{Specifications of the Apertures}
	\label{tab:Specifications_Apertures}
\end{table}

\subsection{Measurements}
\label{subsec:Measurements}
The performed measurements are almost identical, no matter what kind of aperture it is. This is due to the fact, that no lenses were used during this experiment (see section \ref{subsec:experimental_arrangement}).

The maxima are measured for the slits, the anti-slits and the cross-grids. Whilst for the circular apertures the minima are being measured. This is due to the fact, that it is generally easier to measure the maxima. The problem with circular apertures is, that there just is no easy equation to accurately relate the maxima to an angle.
